The premise behind gaps is that there is are obstacles in the world that
must be avoided by staying {\em within} the gap region and only leaving
by moving across the gap curve.
The gap region is known to be collision-free due to line-of-sight
visibility to all points inside of it and to a non-trivial connected
region on the other side of it.  Based on the construction of $\rgpSet$,
there exists a set of local goals on the other side of the gap (relative
to the robot position) that are line-of-sight visible from any point
within the gap region.  

Let the line-of-sight visible local goal point be $\xLG$ as determined
from the chosen gap $\gap^* \in \rgpSet$.  The attractive potential is 
\begin{equation} \label{eq:dpot}
  \attPot(x) = \dist(x, \xLG) + \distHinge(x, \gap^*),
\end{equation}
where the first distance is to the local goal point and the second is
the hinge distance to the gap curve. The hinge distance is the signed
distance clipped to zero out negative values (a composition of the
Heaviside and signed distance functions). In this case negative
distances to the gap lie on the other side of the gap. Thus, the hinge
distance is positive on the robot side of the gap and  vanishes on the
other side of the gap.  These potentials attract the robot to the gap
curve then through to the local goal.  
%Rather than use the gradient for the flow, we will follow the normalized gradient.

Rather than impose an obstacle avoiding potential, which could
create a fixed point in the resulting vector field, a purely rotational
vector field is created 
\begin{equation} \label{eq:gcirc}
  \small \rotVF(x) = 
     \J e^{-\dist_\theta(x, \pLeft)/\bandGA}  
      \frac{\pLeft  - x}{||\pLeft  - x||} \\
     - \J e^{-\dist_\theta(x, \pRight)/\bandGA} 
       \frac{\pRight - x}{||\pRight - x||},
\end{equation}
where $\J = R(-\pi/2)$ is skew-symmetric, and $\dist_\theta(\cdot,
\cdot)$ is the angular distance. $\pLeft$ and $\pRight$ are
points of the left and right sides of the gaps. The vector fields are two
rotational fields anchored at the left and right gap points.  Figure
\ref{figGapGrad} shows an example circulation vector field. 

\begin{figure}[t]
  \vspace*{0.06in}
  \centering
  \begin{tikzpicture}[inner sep=0pt, outer sep=0pt]
    %\node [anchor= south west] (empty) at (0.05, 0.2)
    %{{\includegraphics[width=0.25\columnwidth]{figures/gapfield_empty.jpg}}};
    %\node [anchor=south west] (circulation) at (0.05 + 0.248\columnwidth, 0.2)

    \node [anchor=south west] (circulation) at (0, 0)
    {{\includegraphics[width=0.3\columnwidth]{../figures/gapfield_rotation.jpg}}};
    \node [anchor=south west,xshift=12pt] (attractor) at (circulation.south east)
    {{\includegraphics[width=0.3\columnwidth]{../figures/gapfield_goalatt.jpg}}};
    \node [anchor=south west,xshift=12pt] (combined) at (attractor.south east)
    {{\includegraphics[width=0.3\columnwidth]{../figures/gapfield_total.jpg}}};

    \node at ($(circulation.east)!0.5!(attractor.west)$) {+};
    \node at ($(attractor.east)!0.5!(combined.west)$) {=};
    %\node[anchor=mid,yshift=0pt] at (empty.south) {\scriptsize gap scenario};
    \node[anchor=mid,yshift=0pt,fill=white] at (circulation.south)
      {\scriptsize circulation};
    \node[anchor=mid,yshift=0pt,fill=white] at (attractor.south)
      {\scriptsize attractor};
    \node[anchor=mid,yshift=0pt,fill=white] at (combined.south)
      {\scriptsize combined};
  \end{tikzpicture}
  \caption{Gap gradient field construction. Red triangle is robot location,
  blue circles are gap curve endpoints,
  and green circle is goal point.  \label{figGapGrad}}
  \vspace*{-0.75em}
\end{figure}

