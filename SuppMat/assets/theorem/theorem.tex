\documentclass[letterpaper, 10pt, twocolumn]{article}
\usepackage{multicol}
\usepackage{blindtext}
\usepackage{geometry}
\geometry{letterpaper, margin=0.75in}

\usepackage{xcolor}

\usepackage{tikz}
\usetikzlibrary{calc}


\usepackage{amsmath, amsthm, amsfonts}
\usepackage[utf8]{inputenc}
\usepackage[english]{babel}
\newtheorem{theorem}{Theorem}
\newtheorem{corollary}{Corollary}[theorem]
\newtheorem{lemma}[theorem]{Lemma}
\theoremstyle{definition}
\newtheorem{definition}{Definition}[section]


\DeclareMathOperator{\dist}{d}
\DeclareMathOperator{\proj}{Proj}
\DeclareMathOperator{\atan}{atan}
\usepackage{../potentialgap}

\newcommand{\editadd}[1]{\textcolor{magenta}{#1}}
\newcommand{\modified}[1]{\textcolor{blue}{#1}}
\newcommand{\remove}[1]{\textcolor{yellow}{\st{#1}}}


\begin{document}
\section{Potential Gap Theorem}

\begin{theorem} \label{theTheorem}
  Assume that a point-mass robot with a planar, first-order, fully controlled
  motion model and a 360$^\circ$ scanner operates in a planar environment.
  Let the latest scan measurement from the robot provide a set of
  candidate gaps $\rgpSet$ to travel through based on the {\em Swept Gap
  Prioritization} algorithm followed by gap convexification.  The
  trajectory $\traj$, generated by forward integrating along the vector
  field $\Dv$ constructed from the gap $\gap^* \in \gapSet$ when starting 
  from the robot position $x$ located at the source position of the
  scan, is guaranteed to result in collision-free passage across the gap
  boundary.
\end{theorem}



For this document to be standalone, the important constructions
associated to the motion vector field will be reproduced here, followed
by the proof proper.

\subsection{Potential Gap Gradient Fields}
The premise behind gaps is that there is are obstacles in the world that
must be avoided by staying {\em within} the gap region and only leaving
by moving across the gap curve.
The gap region is known to be collision-free due to line-of-sight
visibility to all points inside of it and to a non-trivial connected
region on the other side of it.  Based on the construction of $\rgpSet$,
there exists a set of local goals on the other side of the gap (relative
to the robot position) that are line-of-sight visible from any point
within the gap region.  

Let the line-of-sight visible local goal point be $\xLG$ as determined
from the chosen gap $\gap^* \in \rgpSet$.  The attractive potential is 
\begin{equation} \label{eq:dpot}
  \attPot(x) = \dist(x, \xLG) + \distHinge(x, \gap^*),
\end{equation}
where the first distance is to the local goal point and the second is
the hinge distance to the gap curve. The hinge distance is the signed
distance clipped to zero out negative values (a composition of the
Heaviside and signed distance functions). In this case negative
distances to the gap lie on the other side of the gap. Thus, the hinge
distance is positive on the robot side of the gap and  vanishes on the
other side of the gap.  These potentials attract the robot to the gap
curve then through to the local goal.  
%Rather than use the gradient for the flow, we will follow the normalized gradient.

Rather than impose an obstacle avoiding potential, which could
create a fixed point in the resulting vector field, a purely rotational
vector field is created 
\begin{equation} \label{eq:gcirc}
  \small \rotVF(x) = 
     \J e^{-\dist_\theta(x, \pLeft)/\bandGA}  
      \frac{\pLeft  - x}{||\pLeft  - x||} \\
     - \J e^{-\dist_\theta(x, \pRight)/\bandGA} 
       \frac{\pRight - x}{||\pRight - x||},
\end{equation}
where $\J = R(-\pi/2)$ is skew-symmetric, and $\dist_\theta(\cdot,
\cdot)$ is the angular distance. $\pLeft$ and $\pRight$ are
points of the left and right sides of the gaps. The vector fields are two
rotational fields anchored at the left and right gap points.  Figure
\ref{figGapGrad} shows an example circulation vector field. 

\begin{figure}[t]
  \vspace*{0.06in}
  \centering
  \begin{tikzpicture}[inner sep=0pt, outer sep=0pt]
    %\node [anchor= south west] (empty) at (0.05, 0.2)
    %{{\includegraphics[width=0.25\columnwidth]{figures/gapfield_empty.jpg}}};
    %\node [anchor=south west] (circulation) at (0.05 + 0.248\columnwidth, 0.2)

    \node [anchor=south west] (circulation) at (0, 0)
    {{\includegraphics[width=0.3\columnwidth]{../figures/gapfield_rotation.jpg}}};
    \node [anchor=south west,xshift=12pt] (attractor) at (circulation.south east)
    {{\includegraphics[width=0.3\columnwidth]{../figures/gapfield_goalatt.jpg}}};
    \node [anchor=south west,xshift=12pt] (combined) at (attractor.south east)
    {{\includegraphics[width=0.3\columnwidth]{../figures/gapfield_total.jpg}}};

    \node at ($(circulation.east)!0.5!(attractor.west)$) {+};
    \node at ($(attractor.east)!0.5!(combined.west)$) {=};
    %\node[anchor=mid,yshift=0pt] at (empty.south) {\scriptsize gap scenario};
    \node[anchor=mid,yshift=0pt,fill=white] at (circulation.south)
      {\scriptsize circulation};
    \node[anchor=mid,yshift=0pt,fill=white] at (attractor.south)
      {\scriptsize attractor};
    \node[anchor=mid,yshift=0pt,fill=white] at (combined.south)
      {\scriptsize combined};
  \end{tikzpicture}
  \caption{Gap gradient field construction. Red triangle is robot location,
  blue circles are gap curve endpoints,
  and green circle is goal point.  \label{figGapGrad}}
  \vspace*{-0.75em}
\end{figure}



\begin{definition}[Gap Gradient Field]
  The potential field $\attPot(\xv)$ and circulation field $\rotVF(\xv)$
  given by \eqref{eq:dpot} \eqref{eq:gcirc} define the gradient field
  $\Dv(\xv) = \hat{\nabla}\attPot(\xv) + \rotVF(\xv)$.
\end{definition}

\begin{definition}
  $\hat \nabla F$ is the gradient of $F$ with normalization to unit length
  for non-zero magnitude gradients.
\end{definition}

\subsection{Potential Gap Proof of Passage}

\begin{proof}[Proof of Theorem \ref{theTheorem}]
Proving that passage through the gap for the robot must happen involves
showing that the boundary of the gap navigation region points inwards
along the robot-to-gap-endpoint edges (or simply {\em gap sides}) and
that there is no fixed point interior to the region.  The only
reasonable flow for any point in the region is to exit via the gap
curve, e.g., {\em gap passage}. 

Along the gap edges, the gradients of \eqref{eq:dpot} either point
inwards or parallel to it, never out by virtue of gap convexity, the
location of the local goal relative to the gap, and by definition of the
gap region. Thus, what must be shown is that the
circulation components also point inwards. On the gap edge, one
circulation term has $\dist_\theta$ vanishing; the circulation is purely
perpendicular and inward pointing. Let this vector be $e_\perp$. Let the
other circulation term contribute the vector $f_\phi$. It satisfies
\begin{equation}
  (e_\perp + f_\phi)\cdot e_\perp = (1 - \cos(\pi - \phi)) \ge 0,
  \ \text{for some}\ \phi > 0,
\end{equation}
which means that $\rotVF(\cdot)$ restricted to gap edges is inward
pointing.  The only outward flow can be on the gap curve.

A similar argument as above applies to show that the interior gap region
points have a non-trivial outward pointing flow, which means that there
cannot be a fixed point for $\Dv$ interior to the gap region. 
Define $e_\rho(x)$ to be the radially directed outward vector for a
point $x \in \gap$. The following properties hold:
\begin{equation}
  e_\rho(x) \cdot \rotVF(x) \ge 0, \quad \text{and} \quad
  e_\rho(x) \cdot \hat \nabla \attPot(x) > 0
\end{equation}
by virtue of the gap angular extent $\alpha_g$ being $90^\circ$; this
angle is the polar space angular difference between the two gap points
relative to the robot reference frame. 
Since the definition of $\hat \nabla$ is that it computes the gradient in
\eqref{eq:dpot} then makes it unit length when non-zero, the gradients
should not vanish.  Furthermore, as the vector field interior to the gap
region always has positive outward pointing contributions
(and therefore no vanishing gradients), there can be no fixed points. 
Positivity implies that any initial point in $\gap^*$ will flow out
through the gap curve, see vector field in Figure \ref{figGapGrad}.  All
planar trajectories starting in the polar triangle defined by $\gap^*$ and
the robot location. and following the constructed vector field are
guaranteed to exit the gap region through the {\em gap curve}, to be
attracted to the local goal, and to be non-colliding. 

The potential gap algorithm generates the trajectory $\traj$ by starting
at the current robot position and following the flow field until passage
through the gap to the local goal point.

\end{proof}
%
%Since there is no sink in the region, and that all edges have inward pointing
%gradient field except the outgoing gap curve, a trajectory greedily following
%the field must flow out through the gap curve, and therefore passage. 
%


\end{document}

