Proving that passage through the gap for the robot must happen involves
showing that the boundary of the gap navigation region points inwards
along the robot-to-gap-endpoint edges (or simply {\em gap sides}) and
that there is no fixed point interior to the region.  The only
reasonable flow for any point in the region is to exit via the gap
curve, e.g., {\em gap passage}. 

Along the gap edges, the gradients of \eqref{eq:dpot} either point
inwards or parallel to it, never out by virtue of gap convexity, the
location of the local goal relative to the gap, and by definition of the
gap region. Thus, what must be shown is that the
circulation components also point inwards. On the gap edge, one
circulation term has $\dist_\theta$ vanishing; the circulation is purely
perpendicular and inward pointing. Let this vector be $e_\perp$. Let the
other circulation term contribute the vector $f_\phi$. It satisfies
\begin{equation}
  (e_\perp + f_\phi)\cdot e_\perp = (1 - \cos(\pi - \phi)) \ge 0,
  \ \text{for some}\ \phi > 0,
\end{equation}
which means that $\rotVF(\cdot)$ restricted to gap edges is inward
pointing.  The only outward flow can be on the gap curve.

A similar argument as above applies to show that the interior gap region
points have a non-trivial outward pointing flow, which means that there
cannot be a fixed point for $\Dv$ interior to the gap region. 
Define $e_\rho(x)$ to be the radially directed outward vector for a
point $x \in \gap$. The following properties hold:
\begin{equation}
  e_\rho(x) \cdot \rotVF(x) \ge 0, \quad \text{and} \quad
  e_\rho(x) \cdot \hat \nabla \attPot(x) > 0
\end{equation}
by virtue of the gap angular extent $\alpha_g$ being $90^\circ$; this
angle is the polar space angular difference between the two gap points
relative to the robot reference frame. 
Since the definition of $\hat \nabla$ is that it computes the gradient in
\eqref{eq:dpot} then makes it unit length when non-zero, the gradients
should not vanish.  Furthermore, as the vector field interior to the gap
region always has positive outward pointing contributions
(and therefore no vanishing gradients), there can be no fixed points. 
Positivity implies that any initial point in $\gap^*$ will flow out
through the gap curve, see vector field in Figure \ref{figGapGrad}.  All
planar trajectories starting in the polar triangle defined by $\gap^*$ and
the robot location. and following the constructed vector field are
guaranteed to exit the gap region through the {\em gap curve}, to be
attracted to the local goal, and to be non-colliding. 

The potential gap algorithm generates the trajectory $\traj$ by starting
at the current robot position and following the flow field until passage
through the gap to the local goal point.
